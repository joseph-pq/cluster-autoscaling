\begin{abstract}
This study evaluates the performance of Horizontal Pod Autoscaling (HPA) in Kubernetes environments using Minikube and Amazon Elastic Kubernetes Service (EKS).
Kubernetes is an open-source platform that enables the management and scaling of containerized applications.
HPA is a key feature that dynamically adjusts the number of pod replicas to handle fluctuating workloads.
The study compares the autoscaling behavior on Minikube, a local Kubernetes setup, with that on AWS EKS, a cloud-based service, focusing on their ability to scale resources in response to varying CPU loads.
Experiments involved deploying a PHP-Apache server and applying artificial load to observe HPA's effectiveness.
Results indicate that both environments successfully scale resources, but with variations in scalability limits, highlighting the trade-offs between local and cloud-based Kubernetes implementations.
\end{abstract}

% Note that keywords are not normally used for peerreview papers.
\begin{IEEEkeywords}
Kubernetes, Horizontal Pod Autoscaler, Minikube, Amazon EKS,
Automatic Scaling, Microservices, Cloud Computing.
\end{IEEEkeywords}

% For peer review papers, you can put extra information on the cover
% page as needed:
% \ifCLASSOPTIONpeerreview
% \begin{center} \bfseries EDICS Category: 3-BBND \end{center}
% \fi
%
% For peerreview papers, this IEEEtran command inserts a page break and
% creates the second title. It will be ignored for other modes.
\IEEEpeerreviewmaketitle
